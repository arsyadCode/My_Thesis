Penyebab Gas Rumah Kaca (GRK) menjadi isu global yang mendesak untuk ditangani, termasuk di Indonesia. Penelitian ini bertujuan untuk mengidentifikasi faktor penyebab GRK di Indonesia melalui analisis data abstrak dari studi literatur lima tahun terakhir. Pendekatan yang digunakan melibatkan teknik \textit{text mining}, \textit{clustering}, dan seleksi menggunakan teori Luhn. Dengan menggunakan algoritma K-Means dalam teknik \textit{clustering}, data abstrak dikategorikan menjadi beberapa klaster berdasarkan kesamaan konten. Metode seleksi menggunakan teori Luhn juga mampu mengidentifikasi kata-kata kunci yang paling relevan dan mewakili relevansi GRK di Indonesia. Hasil analisis menunjukkan bahwa faktor-faktor pada aspek yang memiliki relevansi GRK di Indonesia meliputi diantaranya deforestasi, biodiesel, CO2/karbon, dan listrik. Pada penelitian ini juga memberikan pemahaman yang cukup tentang faktor-faktor yang berhubungan pada aspek GRK di Indonesia dan memberikan informasi penting bagi pembuat kebijakan, para ahli, para pembaca untuk mengembangkan strategi mitigasi yang efektif.

Kata kunci: faktor, gas rumah kaca, kata kunci, nlp, teori luhn, \textit{text mining}