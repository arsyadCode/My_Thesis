The causes of Greenhouse Gas (GHG) emissions have become a pressing global issue to address, including in Indonesia. This study aims to identify the factors contributing to GHG emissions in Indonesia through the analysis of abstract data from literature studies of the past five years. The approach involves techniques such as text mining, clustering, and selection using Luhn's theory. By employing the K-Means algorithm in clustering, abstract data is categorized into several clusters based on content similarity. The Luhn's theory-based selection method also identifies the most relevant keywords that represent the relevance of GHG emissions in Indonesia. The analysis results indicate that factors related to the aspects relevant to GHG emissions in Indonesia include deforestation, biodiesel, CO2/carbon, and electricity. This study provides a comprehensive understanding of the factors associated with the GHG aspect in Indonesia, offering essential insights for policymakers, experts, and readers to develop effective mitigation strategies.

Keywords: factors, greenhouse gas, keywords, luhn's theory, nlp, text mining