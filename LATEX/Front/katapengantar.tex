Dengan memanjatkan segala puji syukur kehadirat Allah SWT yang telah melimpahkan rahmat, taufik, dan hidayah-Nya sehingga penulis dapat menyelesaikan skripsi berupa laporan Tugas Akhir (TA) yang berjudul \textbf{“Analisis Kata Kunci yang Relevan dengan Faktor Gas Rumah Kaca di Indonesia Menggunakan \textit{Text Mining} dan Teori Luhn”}, sebagai salah satu persyaratan dalam menyelesaikan masa belajar di Program Studi S1 Ilmu Komputer, Fakultas Sains dan Ilmu Komputer Universitas Pertamina.
Penulis menyadari bahwa laporan TA ini tidak akan terselesaikan tanpa adanya dukungan, bimbingan, bantuan, dan nasihat dari berbagai pihak selama penyusunan laporan TA ini. Pada kesempatan ini, penulis ingin menyampaikan terima kasih setulus-tulusnya kepada:
\begin{enumerate}
    \item Bapak Ade Irawan, Ph.D selaku Ketua Program Studi Ilmu Komputer Universitas Pertamina dan juga sebagai dosen penguji pertama yang telah memberikan masukan dan saran pada skripsi ini.
    \item Bapak Rangga Ganzar Noegraha, Ph.D selaku dosen penguji kedua yang telah memberikan masukan dan saran pada skripsi ini.
    \item Ibu Dr. Tasmi, S.Si, M.Si. dan Ibu Dr. Ariana Yunita selaku dosen pembimbing pertama dan dosen pembimbing kedua pada skripsi ini yang telah memberikan bimbingan, motivasi, dan arahan selama penyusunan laporan tugas akhir dari awal hingga akhir saat ini.
    \item Seluruh staf pengajar yang telah memberikan ilmu pengetahuan yang tak ternilai selama penulis menempuh pendidikan di Universitas Pertamina.
    \item Seluruh \textit{civitas academica} di lingkungan Universitas Pertamina yang telah menunjang layanan dan dukungan selama proses pembelajaran di Universitas Pertamina.
    \item Keluarga, saudara, dan teman-teman penulis yang telah memberikan doa, kasih sayang, motivasi, dan dorongan di setiap langkah perjuangan penulis.
\end{enumerate}

Penulis menyadari bahwa laporan TA ini belum sempurna dan banyak kekurangan, oleh karena itu, segala saran dan kritik yang membangun diharapkan oleh penulis sebagai penyempurna penulisan laporan ini di masa yang mendatang. Semoga laporan TA ini dapat memberikan manfaat bagi penulis dan para pembaca.
	\vskip 1cm
	\hspace{10cm} Jakarta, 31 Juli 2023 \par
	\vskip 1.75cm
	\hspace{10cm} Kiagus Muhammad Arsyad

