\Appendix{Lampiran 2. Frasa dari Kata Kunci yang Didapatkan}
\label{lampiran2}

Hasil dari kata-kata kunci dengan adanya hasil dari klasterisasi dan teori Luhn, berikut ini berbagai frasa dengan rentang 1-2 kata setiap frasanya. Untuk membantu hal tersebut, digunakan \textit{library Keybert} yang mampu mengidentifikasi kata-kata yang mencerminkan makna dari teks yang dipakai, dalam hal ini dipakai teks dari hasil klasterisasi. 
\begin{figure}[H]
    \centering
    \includegraphics[width=\linewidth]{img/L2-1.png}
    \caption{Hasil frasa yang dapat dibentuk (1)}
    \label{fig:L2-1}
\end{figure}
\begin{figure}[H]
    \centering
    \includegraphics[width=\linewidth]{img/L2-2.png}
    \caption{Hasil frasa yang dapat dibentuk (2)}
    \label{fig:L2-2}
\end{figure}

Adapun setiap elemennya, terdapat sebuah tupel yang berisi frasa dan nilai desimal, pada elemen kedua atau nilai desimal tersebut menunjukkan tingkat kepentingan atau relevansi frasa tersebut dalam teks dalam rentang nilainya 0 sampai 1. Semakin tinggi nilai desimal, semakin penting dan relevan frasa tersebut dalam konteks teks yang sedang dianalisis.