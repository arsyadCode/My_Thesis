\chapter{KESIMPULAN DAN SARAN}
\label{BAB5:kesimpulan}

\section{Kesimpulan}
Dari hasil analisis data teks menggunakan pengolahan dan teknik dengan \textit{text mining}, \textit{clustering}, dan penyeleksian menggunakan teori Luhn pada data abstrak dari studi literatur lima tahun terakhir, telah didapatkan faktor dan subfaktor kata-kata kunci yang memiliki relevan sebagai faktor penyebab Gas Rumah Kaca (GRK) di Indonesia berupa empat kelompok, yaitu \textit{deforestation, biodiesel, co2/carbon, electricity}. Hasil analisis ini dapat memberikan informasi yang bermanfaat terkait permasalahan GRK saat ini dan dapat menjadi landasan untuk langkah-langkah penanganan dan mitigasi lebih lanjut terhadap dampak GRK di Indonesia terutama pada beberapa aspek yang telah dihasilkan dari keempat kelompok hasil \textit{clustering} sebelumnya.

\section{Saran}
Untuk penelitian selanjutnya, disarankan agar dapat memperluas sumber data studi literatur dengan mencakup lebih banyak publikasi terkini dan diversifikasi sumber literatur, termasuk jurnal ilmiah, laporan pemerintah, dan publikasi akademis lainnya. Penggunaan sumber data yang lebih luas akan memberikan gambaran yang lebih komprehensif terkait faktor penyebab GRK di Indonesia.

Selanjutnya, direkomendasikan untuk melanjutkan penelitian dengan kolaborasi para ahli dalam membantu memvalidasi apa yang telah dihasilkan pada penelitian di bidang terkait. Selain itu, dalam upaya meningkatkan pemahaman yang lebih holistik pada aspek dampak GRK di Indonesia, disarankan untuk dilakukan kajian lebih lanjut dengan mempertimbangkan aspek waktu, geografis, dan perubahan iklim, sehingga dapat memberikan pemahaman yang lebih holistik mengenai dampak GRK di Indonesia yang lebih signifikan dalam pengembangan strategi dan kebijakan mitigasi GRK yang efektif serta berkelanjutan di Indonesia. 